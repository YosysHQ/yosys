
\section{Introduction}

\begin{frame}
\sectionpage
\end{frame}

%%%%%%%%%%%%%%%%%%%%%%%%%%%%%%%%%%%%%%%%%%%%%%%%%%%%%%%%%%%%%%%%%%%%%%%%%%%%%

\subsection{Representations of (digital) Circuits}

\begin{frame}[t]{\subsecname}
\begin{itemize}
	\item Graphical
		\begin{itemize}
			\item \alert<1>{Schematic Diagram}
			\item \alert<2>{Physical Layout}
		\end{itemize}
	\bigskip
	\item Non-graphical
		\begin{itemize}
			\item \alert<3>{Netlists}
			\item \alert<4>{Hardware Description Languages (HDLs)}
		\end{itemize}
\end{itemize}
\bigskip
\begin{block}{Definition:
\only<1>{Schematic Diagram}%
\only<2>{Physical Layout}%
\only<3>{Netlists}%
\only<4>{Hardware Description Languages (HDLs)}}
\only<1>{
	Graphical representation of the circtuit topology. Circuit elements
	are represented by symbols and electrical connections by lines. The gometric
	layout is for readability only.
}%
\only<2>{
	The actual physical geometry of the device (PCB or ASIC manufracturing masks).
	This is the final product of the design process.
}%
\only<3>{
	A list of circuit elements and a list of connections. This is the raw circuit
	topology.
}%
\only<4>{
	Computer languages (like programming languages) that can be used to describe
	circuits. HDLs are much more powerful in describing huge circuits than
	schematic diagrams.
}%
\end{block}
\end{frame}

%%%%%%%%%%%%%%%%%%%%%%%%%%%%%%%%%%%%%%%%%%%%%%%%%%%%%%%%%%%%%%%%%%%%%%%%%%%%%

\subsection{Levels of Abstraction for Digital Circuits}

\begin{frame}[t]{\subsecname}
\begin{itemize}
	\item \alert<1>{System Level}
	\item \alert<2>{High Level}
	\item \alert<3>{Behavioral Level}
	\item \alert<4>{Register-Transfer Level (RTL)}
	\item \alert<5>{Logical Gate Level}
	\item \alert<6>{Physical Gate Level}
	\item \alert<7>{Switch Level}
\end{itemize}
\bigskip
\begin{block}{Definition:
\only<1>{System Level}%
\only<2>{High Level}%
\only<3>{Behavioral Level}%
\only<4>{Register-Transfer Level (RTL)}%
\only<5>{Logical Gate Level}%
\only<6>{Physical Gate Level}%
\only<7>{Switch Level}}
\only<1>{
	Overall view of the circuit: E.g. block-diagrams or instruction-set architecture descriptions
}%
\only<2>{
	Functional implementation of circuit in high-level programming language (C, C++, SystemC, Matlab, Python, etc.).
}%
\only<3>{
	Cycle-accurate description of circuit in hardware description language (Verilog, VHDL, etc.).
}%
\only<4>{
	List of registers (flip-flops) and logic functions that calculate the next state from the previous one. Usually
	a netlist utilizing high-level cells such as adders, multiplieres, multiplexer, etc.
}%
\only<5>{
	Netlist of single-bit registers and basic logic gates (such as AND, OR,
	NOT, etc.). Popular form: And-Inverter-Graphs (AIGs) with pairs of primary
	inputs and outputs for each register bit.
}%
\only<6>{
	Netlist of cells that actually are available on the target architecture
	(such as CMOS gates in an ASCI or LUTs in an FPGA). Optimized for
	area and/or and/or speed (static timing or number of logic levels).
}%
\only<7>{
	Netlist of individual transistors.
}%
\end{block}
\end{frame}

%%%%%%%%%%%%%%%%%%%%%%%%%%%%%%%%%%%%%%%%%%%%%%%%%%%%%%%%%%%%%%%%%%%%%%%%%%%%%

\subsection{Digital Circuit Synthesis}

\begin{frame}{\subsecname}
	Synthesis Tools (such as Yosys) can transform HDL code to circuits:

	\bigskip
	\begin{center}
	\begin{tikzpicture}[scale=0.8, every node/.style={transform shape}]
			\tikzstyle{lvl} = [draw, fill=MyBlue, rectangle, minimum height=2em, minimum width=15em]
			\node[lvl] (sys) {System Level};
			\node[lvl] (hl) [below of=sys] {High Level};
			\node[lvl] (beh) [below of=hl] {Behavioral Level};
			\node[lvl] (rtl) [below of=beh] {Register-Transfer Level (RTL)};
			\node[lvl] (lg) [below of=rtl] {Logical Gate Level};
			\node[lvl] (pg) [below of=lg] {Physical Gate Level};
			\node[lvl] (sw) [below of=pg] {Switch Level};

			\draw[dotted] (sys.east)  -- ++(1,0) coordinate (sysx);
			\draw[dotted] (hl.east)  -- ++(1,0) coordinate (hlx);
			\draw[dotted] (beh.east) -- ++(1,0) coordinate (behx);
			\draw[dotted] (rtl.east) -- ++(1,0) coordinate (rtlx);
			\draw[dotted] (lg.east)  -- ++(1,0) coordinate (lgx);
			\draw[dotted] (pg.east)  -- ++(1,0) coordinate (pgx);
			\draw[dotted] (sw.east)  -- ++(1,0) coordinate (swx);

			\draw[gray,|->] (sysx) -- node[right] {System Design} (hlx);
			\draw[|->|] (hlx) -- node[right] {High Level Synthesis (HLS)} (behx);
			\draw[->|] (behx) -- node[right] {Behavioral Synthesis} (rtlx);
			\draw[->|] (rtlx) -- node[right] {RTL Synthesis} (lgx);
			\draw[->|] (lgx) -- node[right] {Logic Synthesis} (pgx);
			\draw[gray,->|] (pgx) -- node[right] {Cell Library} (swx);

			\draw[dotted] (behx) -- ++(4,0) coordinate (a);
			\draw[dotted] (pgx) -- ++(4,0) coordinate (b);
			\draw[|->|] (a) -- node[right] {Yosys} (b);
	\end{tikzpicture}
	\end{center}
\end{frame}

%%%%%%%%%%%%%%%%%%%%%%%%%%%%%%%%%%%%%%%%%%%%%%%%%%%%%%%%%%%%%%%%%%%%%%%%%%%%%

\subsection{What Yosys can and can't do}

\begin{frame}{\subsecname}

Things Yosys can do:
\begin{itemize}
\item Read and process (most of) modern Verilog-2005 code.
\item Perform all kinds of operations on netlist (RTL, Logic, Gate).
\item Perform logic optimiziations and gate mapping with ABC\footnote[frame]{\url{http://www.eecs.berkeley.edu/~alanmi/abc/}}.
\end{itemize}

\bigskip
Things Yosys can't do:
\begin{itemize}
\item Process high-level languages such as C/C++/SystemC.
\item Create physical layouts (place\&route).
\end{itemize}

\bigskip
A typical flow combines Yosys with with a low-level implementation tool, such
as Qflow\footnote[frame]{\url{http://opencircuitdesign.com/qflow/}} for ASIC designs.

\end{frame}

%%%%%%%%%%%%%%%%%%%%%%%%%%%%%%%%%%%%%%%%%%%%%%%%%%%%%%%%%%%%%%%%%%%%%%%%%%%%%

\subsection{Yosys Data- and Control-Flow}

\begin{frame}{\subsecname}
	A (usually short) synthesis script controlls Yosys.

	This scripts contain three types of commands:
	\begin{itemize}
	\item {\bf Frontends}, that read input files (usually Verilog).
	\item {\bf Passes}, that perform transformation on the design in memory.
	\item {\bf Backends}, that write the design in memory to a file (various formats are available, e.g. Verilog, BLIF, EDIF, SPICE, BTOR, etc.).
	\end{itemize}

	\bigskip
	\begin{center}
	\begin{tikzpicture}[scale=0.6, every node/.style={transform shape}]
		\path (-1.5,3) coordinate (cursor);
		\draw[-latex] ($ (cursor) + (0,-1.5) $) -- ++(1,0);
		\draw[fill=orange!10] ($ (cursor) + (1,-3) $) rectangle node[rotate=90] {Frontend} ++(1,3) coordinate (cursor);
		\draw[-latex] ($ (cursor) + (0,-1.5) $) -- ++(1,0);
		\draw[fill=green!10] ($ (cursor) + (1,-3) $) rectangle node[rotate=90] {Pass} ++(1,3) coordinate (cursor);
		\draw[-latex] ($ (cursor) + (0,-1.5) $) -- ++(1,0);
		\draw[fill=green!10] ($ (cursor) + (1,-3) $) rectangle node[rotate=90] {Pass} ++(1,3) coordinate (cursor);
		\draw[-latex] ($ (cursor) + (0,-1.5) $) -- ++(1,0);
		\draw[fill=green!10] ($ (cursor) + (1,-3) $) rectangle node[rotate=90] {Pass} ++(1,3) coordinate (cursor);
		\draw[-latex] ($ (cursor) + (0,-1.5) $) -- ++(1,0);
		\draw[fill=orange!10] ($ (cursor) + (1,-3) $) rectangle node[rotate=90] {Backend} ++(1,3) coordinate (cursor);
		\draw[-latex] ($ (cursor) + (0,-1.5) $) -- ++(1,0);

		\path (-3,-0.5) coordinate (cursor);
		\draw (cursor) -- node[below] {HDL} ++(3,0) coordinate (cursor);
		\draw[|-|] (cursor) -- node[below] {Internal Format (RTLIL)} ++(8,0) coordinate (cursor);
		\draw (cursor) -- node[below] {Netlist} ++(3,0);

		\path (-3,3.5) coordinate (cursor);
		\draw[-] (cursor) -- node[above] {High-Level} ++(3,0) coordinate (cursor);
		\draw[-] (cursor) -- ++(8,0) coordinate (cursor);
		\draw[->] (cursor) -- node[above] {Low-Level} ++(3,0);
	\end{tikzpicture}
	\end{center}
\end{frame}

%%%%%%%%%%%%%%%%%%%%%%%%%%%%%%%%%%%%%%%%%%%%%%%%%%%%%%%%%%%%%%%%%%%%%%%%%%%%%

\subsection{Example Synthesis Script}

\begin{frame}[t]{\subsecname}

\setbeamercolor{alerted text}{fg=white,bg=red}

\begin{minipage}[t]{6cm}
\tt\scriptsize
\# read design\\
\boxalert<1>{read\_verilog mydesign.v}\\
\boxalert<2>{hierarchy -check -top mytop}

\medskip
\# the high-level stuff\\
\boxalert<3>{proc}; \boxalert<4>{opt}; \boxalert<5>{memory}; \boxalert<6>{opt}; \boxalert<7>{fsm}; \boxalert<8>{opt}

\medskip
\# mapping to internal cell library\\
\boxalert<9>{techmap}; \boxalert<10>{opt}

\bigskip
\it continued\dots
\end{minipage}
\begin{minipage}[t]{5cm}
\tt\scriptsize
\# mapping flip-flops to mycells.lib\\
\boxalert<11>{dfflibmap -liberty mycells.lib}

\medskip
\# mapping logic to mycells.lib\\
\boxalert<12>{abc -liberty mycells.lib}

\medskip
\# cleanup\\
\boxalert<13>{clean}

\medskip
\# write synthesized design\\
\boxalert<14>{write\_verilog synth.v}
\end{minipage}

\vskip1cm

\begin{block}{Command: \tt
\only<1>{read\_verilog mydesign.v}%
\only<2>{hierarchy -check -top mytop}%
\only<3>{proc}%
\only<4>{opt}%
\only<5>{memory}%
\only<6>{opt}%
\only<7>{fsm}%
\only<8>{opt}%
\only<9>{techmap}%
\only<10>{opt}%
\only<11>{dfflibmap -liberty mycells.lib}%
\only<12>{abc -liberty mycells.lib}%
\only<13>{clean}%
\only<14>{write\_verilog synth.v}}
\only<1>{
	TBD
}%
\only<2>{
	TBD
}%
\only<3>{
	TBD
}%
\only<4>{
	TBD
}%
\only<5>{
	TBD
}%
\only<6>{
	TBD
}%
\only<7>{
	TBD
}%
\only<8>{
	TBD
}%
\only<9>{
	TBD
}%
\only<10>{
	TBD
}%
\only<11>{
	TBD
}%
\only<12>{
	TBD
}%
\only<13>{
	TBD
}%
\only<14>{
	TBD
}%
\end{block}

\end{frame}

%%%%%%%%%%%%%%%%%%%%%%%%%%%%%%%%%%%%%%%%%%%%%%%%%%%%%%%%%%%%%%%%%%%%%%%%%%%%%

\subsection{Running the Synthesis Script}

\begin{frame}[fragile]{\subsecname{} -- Verilog Source: \tt counter.v}
\lstinputlisting[xleftmargin=1cm, language=Verilog]{PRESENTATION_Intro/counter.v}
\end{frame}

\begin{frame}[fragile]{\subsecname{} -- Cell Library: \tt mycells.lib}
\begin{columns}
\column[t]{5cm}
\lstinputlisting[basicstyle=\ttfamily\fontsize{8pt}{10pt}\selectfont, language=liberty, lastline=20]{PRESENTATION_Intro/mycells.lib}
\column[t]{5cm}
\lstinputlisting[basicstyle=\ttfamily\fontsize{8pt}{10pt}\selectfont, language=liberty, firstline=21]{PRESENTATION_Intro/mycells.lib}
\end{columns}
\end{frame}

\begin{frame}[t, fragile]{\subsecname{} -- Step 1/4}
\begin{verbatim}
read_verilog counter.v
hierarchy -check -top counter
\end{verbatim}

\vfill
\includegraphics[width=\linewidth,trim=0 0cm 0 0cm]{PRESENTATION_Intro/counter_00.pdf}
\end{frame}

\begin{frame}[t, fragile]{\subsecname{} -- Step 2/4}
\begin{verbatim}
proc; opt; memory; opt; fsm; opt
\end{verbatim}

\vfill
\includegraphics[width=\linewidth,trim=0 0cm 0 0cm]{PRESENTATION_Intro/counter_01.pdf}
\end{frame}

\begin{frame}[t, fragile]{\subsecname{} -- Step 3/4}
\begin{verbatim}
techmap; opt
\end{verbatim}

\vfill
\includegraphics[width=\linewidth,trim=0 0cm 0 0cm]{PRESENTATION_Intro/counter_02.pdf}
\end{frame}

\begin{frame}[t, fragile]{\subsecname{} -- Step 4/4}
\begin{verbatim}
dfflibmap -liberty mycells.lib
abc -liberty mycells.lib
clean
\end{verbatim}

\vfill
\includegraphics[width=\linewidth,trim=0 0cm 0 0cm]{PRESENTATION_Intro/counter_03.pdf}
\end{frame}

