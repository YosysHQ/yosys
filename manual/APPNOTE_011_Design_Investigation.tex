
% IEEEtran howto:
% http://ftp.univie.ac.at/packages/tex/macros/latex/contrib/IEEEtran/IEEEtran_HOWTO.pdf
\documentclass[9pt,technote,a4paper]{IEEEtran}

\usepackage[T1]{fontenc}   % required for luximono!
\usepackage[scaled=0.8]{luximono}  % typewriter font with bold face

% To install the luximono font files:
% getnonfreefonts-sys --all        or
% getnonfreefonts-sys luximono
%
% when there are trouble you might need to:
% - Create /etc/texmf/updmap.d/99local-luximono.cfg
%   containing the single line: Map ul9.map
% - Run update-updmap followed by mktexlsr and updmap-sys
%
% This commands must be executed as root with a root environment
% (i.e. run "sudo su" and then execute the commands in the root
% shell, don't just prefix the commands with "sudo").

\usepackage[unicode,bookmarks=false]{hyperref}
\usepackage[english]{babel}
\usepackage[utf8]{inputenc}
\usepackage{amssymb}
\usepackage{amsmath}
\usepackage{amsfonts}
\usepackage{units}
\usepackage{nicefrac}
\usepackage{eurosym}
\usepackage{graphicx}
\usepackage{verbatim}
\usepackage{algpseudocode}
\usepackage{scalefnt}
\usepackage{xspace}
\usepackage{color}
\usepackage{colortbl}
\usepackage{multirow}
\usepackage{hhline}
\usepackage{listings}
\usepackage{float}

\usepackage{tikz}
\usetikzlibrary{calc}
\usetikzlibrary{arrows}
\usetikzlibrary{scopes}
\usetikzlibrary{through}
\usetikzlibrary{shapes.geometric}

\def\FIXME{{\color{red}\bf FIXME}}

\lstset{basicstyle=\ttfamily,frame=trBL,xleftmargin=2em,xrightmargin=1em,numbers=left}

\begin{document}

\title{Yosys Application Note 011: \\ Interactive Design Investigation}
\author{Clifford Wolf \\ November 2013}
\maketitle

\begin{abstract}
Yosys \cite{yosys} can be a great environment for building custom synthesis
flows \cite{glaserwolf}. It can also be an excellent tool for teaching and
learning Verilog based RTL synthesis. In both applications it is of great
importance to be able to analyze the designs produces easily.

This Yosys application note covers the generation of circuit diagrams with the
Yosys {\tt show} command and the selection of interesting parts of the circuit
using the {\tt select} command.
\end{abstract}

\section{Installation and Prerequisites}

This Application Note is based on GIT Rev. {\tt \FIXME} from \FIXME{} of
Yosys \cite{yosys}. The {\tt README} file covers how to install Yosys. The
{\tt show} command requires a working installation of GraphViz \cite{graphviz}
for generating the actual circuit diagrams. Yosys must be build with Qt
support in order to activate the built-in SVG viewer. Alternatively an
external viewer can be used.

\section{Introduction to the {\tt show} command}

\FIXME

\begin{figure}[b]
\begin{lstlisting}
$ cat example.ys
read_verilog example.v
show -pause
proc
show -pause
opt
show -pause

$ cat example.v
module example(input clk, a, b, c,
               output reg [1:0] y);
    always @(posedge clk)
        if (c)
            y <= c ? a + b : 2'd0;
endmodule
\end{lstlisting}
\caption{Synthesis script with added show commands and example code}
\label{example_src}
\end{figure}

\begin{figure}[b]
\includegraphics[width=\linewidth]{APPNOTE_011_Design_Investigation/example_00.pdf}
\includegraphics[width=\linewidth]{APPNOTE_011_Design_Investigation/example_01.pdf}
\includegraphics[width=\linewidth]{APPNOTE_011_Design_Investigation/example_02.pdf}
\caption{\tt Output of the three show commands from Fig.~\ref{example_src}}
\label{example_out}
\end{figure}


\begin{thebibliography}{9}

\bibitem{yosys}
Clifford Wolf. The Yosys Open SYnthesis Suite.
\url{http://www.clifford.at/yosys/}

\bibitem{glaserwolf}
Johann Glaser. Clifford Wolf. Methodology and Example-Driven Interconnect
Synthesis for Designing Heterogeneous Coarse-Grain Reconfigurable
Architectures. In: Jan Haase (Editor). {\it Models, Methods, and Tools for Complex Chip Design.
Lecture Notes in Electrical Engineering. Volume 265, 2014, pp 201-221.\/}
\href{http://dx.doi.org/10.1007/978-3-319-01418-0_12}{DOI 10.1007/978-3-319-01418-0\_12}

\bibitem{graphviz}
Graphviz - Graph Visualization Software.
\url{http://www.graphviz.org/}

\end{thebibliography}

\end{document}
